% Options for packages loaded elsewhere
% Options for packages loaded elsewhere
\PassOptionsToPackage{unicode}{hyperref}
\PassOptionsToPackage{hyphens}{url}
\PassOptionsToPackage{dvipsnames,svgnames,x11names}{xcolor}
%
\documentclass[
  letterpaper,
  DIV=11,
  numbers=noendperiod]{scrartcl}
\usepackage{xcolor}
\usepackage{amsmath,amssymb}
\setcounter{secnumdepth}{-\maxdimen} % remove section numbering
\usepackage{iftex}
\ifPDFTeX
  \usepackage[T1]{fontenc}
  \usepackage[utf8]{inputenc}
  \usepackage{textcomp} % provide euro and other symbols
\else % if luatex or xetex
  \usepackage{unicode-math} % this also loads fontspec
  \defaultfontfeatures{Scale=MatchLowercase}
  \defaultfontfeatures[\rmfamily]{Ligatures=TeX,Scale=1}
\fi
\usepackage{lmodern}
\ifPDFTeX\else
  % xetex/luatex font selection
\fi
% Use upquote if available, for straight quotes in verbatim environments
\IfFileExists{upquote.sty}{\usepackage{upquote}}{}
\IfFileExists{microtype.sty}{% use microtype if available
  \usepackage[]{microtype}
  \UseMicrotypeSet[protrusion]{basicmath} % disable protrusion for tt fonts
}{}
\makeatletter
\@ifundefined{KOMAClassName}{% if non-KOMA class
  \IfFileExists{parskip.sty}{%
    \usepackage{parskip}
  }{% else
    \setlength{\parindent}{0pt}
    \setlength{\parskip}{6pt plus 2pt minus 1pt}}
}{% if KOMA class
  \KOMAoptions{parskip=half}}
\makeatother
% Make \paragraph and \subparagraph free-standing
\makeatletter
\ifx\paragraph\undefined\else
  \let\oldparagraph\paragraph
  \renewcommand{\paragraph}{
    \@ifstar
      \xxxParagraphStar
      \xxxParagraphNoStar
  }
  \newcommand{\xxxParagraphStar}[1]{\oldparagraph*{#1}\mbox{}}
  \newcommand{\xxxParagraphNoStar}[1]{\oldparagraph{#1}\mbox{}}
\fi
\ifx\subparagraph\undefined\else
  \let\oldsubparagraph\subparagraph
  \renewcommand{\subparagraph}{
    \@ifstar
      \xxxSubParagraphStar
      \xxxSubParagraphNoStar
  }
  \newcommand{\xxxSubParagraphStar}[1]{\oldsubparagraph*{#1}\mbox{}}
  \newcommand{\xxxSubParagraphNoStar}[1]{\oldsubparagraph{#1}\mbox{}}
\fi
\makeatother


\usepackage{longtable,booktabs,array}
\usepackage{calc} % for calculating minipage widths
% Correct order of tables after \paragraph or \subparagraph
\usepackage{etoolbox}
\makeatletter
\patchcmd\longtable{\par}{\if@noskipsec\mbox{}\fi\par}{}{}
\makeatother
% Allow footnotes in longtable head/foot
\IfFileExists{footnotehyper.sty}{\usepackage{footnotehyper}}{\usepackage{footnote}}
\makesavenoteenv{longtable}
\usepackage{graphicx}
\makeatletter
\newsavebox\pandoc@box
\newcommand*\pandocbounded[1]{% scales image to fit in text height/width
  \sbox\pandoc@box{#1}%
  \Gscale@div\@tempa{\textheight}{\dimexpr\ht\pandoc@box+\dp\pandoc@box\relax}%
  \Gscale@div\@tempb{\linewidth}{\wd\pandoc@box}%
  \ifdim\@tempb\p@<\@tempa\p@\let\@tempa\@tempb\fi% select the smaller of both
  \ifdim\@tempa\p@<\p@\scalebox{\@tempa}{\usebox\pandoc@box}%
  \else\usebox{\pandoc@box}%
  \fi%
}
% Set default figure placement to htbp
\def\fps@figure{htbp}
\makeatother





\setlength{\emergencystretch}{3em} % prevent overfull lines

\providecommand{\tightlist}{%
  \setlength{\itemsep}{0pt}\setlength{\parskip}{0pt}}



 


\KOMAoption{captions}{tableheading}
\makeatletter
\@ifpackageloaded{caption}{}{\usepackage{caption}}
\AtBeginDocument{%
\ifdefined\contentsname
  \renewcommand*\contentsname{Table of contents}
\else
  \newcommand\contentsname{Table of contents}
\fi
\ifdefined\listfigurename
  \renewcommand*\listfigurename{List of Figures}
\else
  \newcommand\listfigurename{List of Figures}
\fi
\ifdefined\listtablename
  \renewcommand*\listtablename{List of Tables}
\else
  \newcommand\listtablename{List of Tables}
\fi
\ifdefined\figurename
  \renewcommand*\figurename{Figure}
\else
  \newcommand\figurename{Figure}
\fi
\ifdefined\tablename
  \renewcommand*\tablename{Table}
\else
  \newcommand\tablename{Table}
\fi
}
\@ifpackageloaded{float}{}{\usepackage{float}}
\floatstyle{ruled}
\@ifundefined{c@chapter}{\newfloat{codelisting}{h}{lop}}{\newfloat{codelisting}{h}{lop}[chapter]}
\floatname{codelisting}{Listing}
\newcommand*\listoflistings{\listof{codelisting}{List of Listings}}
\makeatother
\makeatletter
\makeatother
\makeatletter
\@ifpackageloaded{caption}{}{\usepackage{caption}}
\@ifpackageloaded{subcaption}{}{\usepackage{subcaption}}
\makeatother
\usepackage{bookmark}
\IfFileExists{xurl.sty}{\usepackage{xurl}}{} % add URL line breaks if available
\urlstyle{same}
\hypersetup{
  pdftitle={Weak Forms and Problem Setup --- Twisting of a Neo-Hookean Block},
  colorlinks=true,
  linkcolor={blue},
  filecolor={Maroon},
  citecolor={Blue},
  urlcolor={Blue},
  pdfcreator={LaTeX via pandoc}}


\title{Weak Forms and Problem Setup --- Twisting of a Neo-Hookean Block}
\author{}
\date{2025-07-01}
\begin{document}
\maketitle


\section{1. Governing equations}\label{governing-equations}

\subsection{1.1 Geometry and boundary}\label{geometry-and-boundary}

The reference configuration is

\[
\Omega = (0, L_x) \times (0, L_y) \times (0, L_z) \subset \mathbb{R}^3, \quad L_x = 10,\ L_y = L_z = 1.
\]

Partition the boundary:

\[
\Gamma_L = \{ x = 0 \}, \quad \Gamma_R = \{ x = L_x \}, \quad \Gamma_N = \partial \Omega \setminus (\Gamma_L \cup \Gamma_R).
\]

\subsection{1.2 Material model}\label{material-model}

Let ( u : \Omega \to \mathbb{R}\^{}3 ) be the displacement. Define:

\[
F = I + \nabla u, \quad J = \det F, \quad F^{-T} = (F^{-1})^T.
\]

For Lamé parameters ( \mu, \lambda \textgreater{} 0 ), the first
Piola--Kirchhoff stress is:

\[
P(F) = \mu F + (\lambda \ln J - \mu) F^{-T}.
\]

\subsection{1.3 Strong form}\label{strong-form}

Seek ( u ) such that:

\[
-\nabla \cdot P(u) = 0 \quad \text{in } \Omega, \\
u = 0 \quad \text{on } \Gamma_L, \\
u = g \quad \text{on } \Gamma_R, \\
P(u)n = 0 \quad \text{on } \Gamma_N,
\]

with prescribed displacement on ( \Gamma\_R ):

\[
g(x, y, z) =
\begin{pmatrix}
-0.1 \\
y(\cos \pi - 1) - z \sin \pi \\
y \sin \pi + z(\cos \pi - 1)
\end{pmatrix}.
\]

\section{2. Weak form and derivations}\label{weak-form-and-derivations}

Multiply the strong form by a test function ( v \in V ) and integrate
over ( \Omega ):

\[
0 = \int_\Omega (-\nabla \cdot P) \cdot v \, dx = -\int_{\partial \Omega} (Pn) \cdot v \, ds + \int_\Omega P : \nabla v \, dx.
\]

On ( \Gamma\_N ), we have ( Pn = 0 ), and on ( \Gamma\_L \cup \Gamma\_R
), the variation ( v = 0 ), so the boundary integral vanishes.

Hence the weak form:

\[
\int_\Omega \left[ \mu\, \text{tr}(F^T \nabla v) + (\lambda \ln J - \mu)\, \text{tr}(F^{-T} \nabla v) \right] dx = 0 \quad \forall v \in V.
\]

\subsubsection{Derivation of the weak
form}\label{derivation-of-the-weak-form}

Starting from:

\[
0 = \int_\Omega (-\nabla \cdot P) \cdot v \, dx,
\]

apply Gauss' theorem and note vanishing boundary terms as above.

Rewriting ( P : \nabla v = \text{tr}(P\^{}T \nabla v) ) and substituting
the expression for ( P(F) ) yields the boxed form.

\subsection{2.1 Jacobian (Fréchet
derivative)}\label{jacobian-fruxe9chet-derivative}

Define the residual functional:

\[
R(u)[v] = \int_\Omega P(F(u)) : \nabla v \, dx.
\]

Its Fréchet derivative in direction ( \delta u ) is:

\[
DR(u)[\delta u, v] =
\left. \frac{d}{d\varepsilon} R(u + \varepsilon \delta u)[v] \right|_{\varepsilon = 0}
= \int_\Omega \delta P : \nabla v \, dx.
\]

We compute variations:

\begin{itemize}
\tightlist
\item
  ( \delta F = \nabla \delta u )
\item
  ( \delta J = J , \text{tr}(F\^{}\{-1\} \nabla \delta u) )
\item
  ( \delta \ln J = \text{tr}(F\^{}\{-1\} \nabla \delta u) )
\item
  ( \delta(F\^{}\{-T\}) = -F\^{}\{-T\} (\nabla \delta u) F\^{}\{-T\} )
\end{itemize}

Thus:

\[
\delta P =
\mu \nabla \delta u
- (\lambda \ln J - \mu) F^{-T} (\nabla \delta u) F^{-T}
+ \lambda \, \text{tr}(F^{-1} \nabla \delta u) F^{-T}.
\]

Substitution gives the Jacobian bilinear form:

\[
DR(u)[\delta u, v] =
\int_\Omega \left[
\mu\, \text{tr}((\nabla \delta u)^T \nabla v)
- (\lambda \ln J - \mu)\, \text{tr}((\nabla \delta u F^{-1})^T (\nabla v F^{-1}))
+ \lambda\, \text{tr}(F^{-1} \nabla \delta u) \, \text{tr}(F^{-1} \nabla v)
\right] dx.
\]




\end{document}
